%\documentclass[brudnopis]{xmgr}
% Jeśli nowe rozdziały mają się zaczynać na stronach nieparzystych:
\documentclass[openright]{xmgr}

% install minted package to highlight source code
%\usepackage{minted}
\usepackage{listings}
\usepackage{float}

% Define Colors
\usepackage{color}
\definecolor{eclipseBlue}{RGB}{42,0.0,255}
\definecolor{eclipseGreen}{RGB}{63,127,95}
\definecolor{eclipsePurple}{RGB}{127,0,85}
\definecolor{eclipseGreem}{RGB}{0,66,0}
\definecolor{eclipseRed}{RGB}{127,20,20}

% Define Language
\lstdefinelanguage{GML}
{
  % list of keywords
  morekeywords=[1]{
    if, else,
    while, exit,
    for
  },
morekeywords=[2]{room_name, room, checked, tasks, i, worse, x, point_direction, direction_s, y, current_hit, hit_count, flood_color, arg_color, n, flood_flooded, win_count, grab, xx, yy, speed_ice, now_turned, now_shown, card_deck, id_fac, delay_time, now_shown_id},
morekeywords=[3]{global, true, false, room_get_name, gms_ini_player_write, floor, random_range, xstart, ystart, image_index, depth, choose, place_meeting, mouse_x, mouse_y, device_get_tilt_x, id, alarm,  },
  sensitive=false, % keywords are not case-sensitive
  morecomment=[l]{//}, % l is for line comment
  morecomment=[s]{/*}{*/}, % s is for start and end delimiter
  morestring=[b]" % defines that strings are enclosed in double quotes
}
 
% Set Language
\lstset{
  language={GML},
  basicstyle=\small\ttfamily, % Global Code Style
  captionpos=b, % Position of the Caption (t for top, b for bottom)
  extendedchars=true, % Allows 256 instead of 128 ASCII characters
  tabsize=2, % number of spaces indented when discovering a tab 
  columns=fixed, % make all characters equal width
  keepspaces=true, % does not ignore spaces to fit width, convert tabs to spaces
  showstringspaces=false, % lets spaces in strings appear as real spaces
  breaklines=true, % wrap lines if they don't fit
  frame=trbl, % draw a frame at the top, right, left and bottom of the listing
  frameround=tttt, % make the frame round at all four corners
  framesep=4pt, % quarter circle size of the round corners
%  numbers=left, % show line numbers at the left
  numberstyle=\tiny\ttfamily, % style of the line numbers
  commentstyle=\color{eclipseGreen}, % style of comments
  keywordstyle=[1]\color{eclipsePurple}, % style of keywords
  keywordstyle=[2]\color{eclipseGreem}, % style of keywords
  keywordstyle=[3]\color{eclipseRed}, % style of keywords
  stringstyle=\color{eclipseBlue}, % style of strings
}



%\defaultfontfeatures{Scale=MatchLowercase}
%\setmainfont[Numbers=OldStyle,Ligatures=TeX]{Minion Pro}
%\setsansfont[Numbers=OldStyle,Ligatures=TeX]{Myriad Pro}
% for fontspec version < 2.0
% \setmainfont[Numbers=OldStyle,Mapping=tex-text]{Minion Pro}
% \setsansfont[Numbers=OldStyle,Mapping=tex-text]{Myriad Pro}
%\setmonofont[Scale=0.75]{Monaco}

% Opcjonalnie identyfikator dokumentu
% drukowany tylko z włączoną opcją 'brudnopis':
\wersja   {wersja wstępna [\ymdtoday]}

\author   {Marcin Szpaderski}
\nralbumu {195\,008}
\email    {szpaderski.marcin@gmail.coml}


\title    {Aplikacja wspierająca badania nad efektem Zeigarnik}
\date     {2017}
\miejsce  {Gdańsk}

\opiekun  {dr Włodzimierz Bzyl}

% dodatkowe polecenia
\renewcommand{\filename}[1]{\texttt{#1}}
%\definecolor{stress}{cmyk}{0,1,0.13,0} % RubineRed
%\definecolor{topic}{cmyk}{0.98,0.13,0,0.43} % MidnightBlue

\begin{document}

% streszczenie
\begin{abstract}
W ramach pracy napisano aplikację na urządzenia mobilne o nazwie ”Pułapka
Zeigarnik”. Do stworzenia aplikacji wykorzystane zostało środowisko do
projektowania gier i programów komputerowych ”GameMaker: Studio”, który
oferuje między innymi unikalny język skryptowy GML. Napisano osiem
testów, które użytkownik ma za zadanie rozwiązać w określonym czasie.

Zaimplementowane zostało również przesyłanie danych z testów i ankiet
na zewnętrzny serwer do dalszego przetworzenia. Użyte zostało rozszerzenie
GameMaker Server, za pomocą którego dane są przechowywane w plikach o
rozszerzeniu .ini. Dla każdego użytkownika tworzy się osobny plik.

Większość grafik potrzebnych do funkcjonowania aplikacji została wykonana
przeze mnie w programie Paint.NET. Wszystkie zapożyczone grafiki
zostały uwzględnione w bibliografii.

Aplikacja została wdrożona i umieszczona w repozytorium GitHub. Jest
dostępna do pobrania ze strony: \url{https://tinyurl.com/ydd98sny} . Kod
źródłowy całej aplikacji także znajduje się w tym repozytorium i jest dostępny
pod adresem: \url{https://tinyurl.com/ycrsr5uu}.

Dla ułatwienia dostępu do aplikacji przygotowana została wersja testowa,
możliwa do zainstalowania na urządzeniu z systemem Android jak i
Windows. Jako że aplikacja jest przystosowana do systemu Android, używanie
jej pod systemem Windows może nie dawać oczekiwanych rezultatów.
W pliku README.md znajdują się wymagane do instalacji informacje.
\url{https://tinyurl.com/y8t733ev} .

W zaprojektowaniu testów brał udział magister psychologii Bartosz Tomkiewicz.
Po zaimplementowaniu i wdrożeniu aplikacji, została ona przedstawiona
grupie osób, które miały wykonać test. Wyniki zostały zanalizowane i
przedstawione na końcu pracy.






\end{abstract}

% słowa kluczowe
\keywords{pamięć ludzka, efekt Zeigarnik, aplikacje mobilne, środowisko programistyczne GameMaker, GameMaker Server}

% tytuł i spis treści
\maketitle

% wstęp
\introduction

Zamiar niekoniecznie oznacza z góry określoną okazję do jego zrealizowania,
lecz potrzebę lub tymczasową potrzebę, która stwarza taką okazję. Bluma
Zeigarnik długo zastanawiała się nad tym stwierdzeniem, próbując zbadać,
jak wywołać u człowieka tę chwilową potrzebę, która wpływa na naszą
pamięć. Celem niniejszej pracy jest stworzenie aplikacji, która będzie pełnić
funkcję pomocniczą przy przeprowadzaniu badań nad tym, co dziś nazywamy
Efektem Zeigarnik.

Efekt Zeigarnik został nazwany i opisany przez Blumę W. Zeigarnik\footnote{Bluma W. Zeigarnik (09.11.1900 - 24.02.1988) - rosyjska psycholog i psychiatra.} w 1927 roku. Objaśnia on pojęcie psychologiczne związane z zagadnieniami pamięci
i motywacji psychologii ogólnej oraz w rezultacie wykazuje, że ludziom łatwiej jest przypomnieć sobie czynności, które zostały im przerwane niż te, które wykonali bez żadnych problemów do końca. Przykładem tego efektu i jednocześnie tym, co zainspirowało Blumę Zeigarnik do badań są kelnerki w restauracjach, które potrafią zapamiętać na raz zamówienia nawet kilku obsługiwanych w danym momencie osób, ale mają znaczne problemy z przypomnieniem sobie klientów, którzy opuścili już lokal. Pomysł na aplikację, która pomaga badać ten efekt pojawił się podczas rozmowy ze znajomym, Bartoszem Tomkiewiczem, który ukończył studia magisterskie na kierunku psychologia i który postanowił pomóc mi z przygotowaniem zadań oraz testów, które najlepiej będą nadawać się do badania Efektu Zeigarnik. Uczestniczył również w analizie wyników uzyskanych przy pomocy aplikacji. 

W części teoretycznej pracy zostaną opisane główne założenia przyjęte podczas
projektowania aplikacji, sposoby rozwiązania konkretnych związanych z nimi problemów oraz możliwości samej aplikacji w zakresie przetwarzania informacji dostarczanych jej przez użytkowników.

\begin{figure}[H]
\centering
\includegraphics[width=6cm]{images/app}
\caption{Wizualizacja: Przykładowy wygląd testu w aplikacji!}
\label{fig:app}
\end{figure}

Projekt został wykonany na zasadzie aplikacji mobilnej. Aplikacja została
zaprojektowana i wykonana w środowisku The GameMaker: Studio, które
wykorzystuje unikalny dla siebie język GML, składnią zbliżony do C++ lub
Pascal. Wybrałem takie środowisko ze względu na chęć rozszerzenia swojej
wiedzy z zakresu implementacji programów i gier w środowiskach do tego
przystosowanych. Wymieniona technologia zostanie opisana w niniejszej
pracy, przedstawione będą zalety jej wykorzystania oraz szczegóły implementacji.

\chapter{Założenia aplikacji}
Głównym celem aplikacji jest stworzenie bazy informacji w celu przeprowadzenia
badań nad Efektem Zeigarnik. Test jest jednorazowy, od użytkownika
wymaga się, aby podszedł do niego tylko raz oraz bez wiedzy, co tak naprawdę
wykonuje.

Założeniem badania jest sprawdzenie, czy uczestnik po wykonaniu wszystkich
zadań będzie w stanie łatwiej przypomnieć sobie te, które wykonał do
końca, czy te, które zostały mu przerwane przed zakończeniem.

Po wykonaniu wszystkich testów użytkownik dostanie do wypełnienia
ankietę, która pozwoli ustalić, co zapamiętał. Wyniki poszczególnych
zadań, jak i treść ankiety wysłane zostaną na serwer.


\section{Porównanie wybranych środowisk programistycznych}
Aplikacja skierowana jest na tablety i inne urządzenia przenośne. Została
stworzona w środowisku programistycznym o nazwie GameMaker: Studio.
Środowisko to wybrałem spośród wielu innych dostępnych na rynku. W tej
chwili najpopularniejszymi środowiskami wydają się być Microsoft Visual
Studio oraz Unity. W tym rozdziale wyznaczę jakie są różnice pomiędzy
nimi a tym które ostatecznie wybrałem oraz postaram się uzasadnić swój wybór.


\subsection{Unity a GameMaker}
Środowisko Unity jest bardzo podobne do wybranego przeze mnie środowiska
GameMaker. Służy głównie do tworzenia trójwymiarowych lub dwuwymiarowych gier komputerowych lub materiałów interaktywnych
takich jak wizualizacje czy animacje. Tak samo jak GameMaker, może być wykorzystany do
tworzenia aplikacji na przeglądarki internetowe, komputery osobiste, konsole
gier wideo oraz wszelkie urządzenia mobilne. Oba środowiska posiadają unikalne
języki skryptowe. Unity wykorzystuje UnityScript, a GameMaker swój
GML. Oba te języki są bazowane na języku JavaScript, do którego są bardzo podobne składnią.

Głównym aspektem, który przekonał mnie do GameMakera był fakt, że Unity jest narzędziem przede wszystkim do pracy
nad elementami 3D, podczas gdy mi zależało na aplikacji w dwóch wymiarach. Nie
jest oczywiście niemożliwe stworzenie dwuwymiarowej aplikacji w Unity, ale, w mojej opinii, lepiej jest korzystać z narzędzi nastawionych przede wszystkim na żądany efekt.


\subsection{Microsoft Visual Studio a GameMaker}
Środowisko Microsoft Visual Studio pozwala na tworzenie aplikacji na wiele
systemów, jednak jest nastawiony przede wszystkim na systemy wspierające
Windowsa. Co za tym idzie, zrobienie aplikacji na
urządzenia mobilne z systemem Android, wymaga niestety dodatkowych frameworków
oraz ograniczeń, których nie byłoby w przypadku programowania
pod Windows Phone. Zaletą GameMakera jest tutaj zdecydowanie możliwość łatwej zmiany aplikacji tak, aby działała pod oboma systemami
bez zbędnych utrudnień.

Podobnie jest z językiem programowania. Pomimo, że teoretycznie Visual Studio przystosowane jest do bardzo wielu języków programowania, domyślnym
i najbardziej dopracowanym językiem jest tam C\# (odróżnieniu od Unity i GameMakera, które skupiają się na języku JavaScript).



\section{Zadania}
Na pierwszą wersję aplikacji zaplanowane jest osiem różnych zadań. W momencie
rozpoczęcia testu wylosowana zostanie połowa testów zaokrąglona w
dół, a czas dany do ich rozwiązania zostanie skrócony tak, by użytkownik
zdążył zaznajomić się z poleceniem lub nawet zacząć je wykonywać,
lecz aby na pewno nie zdążył go ukończyć.

Aplikacja jest przygotowana w taki sposób, że umożliwia stosunkowo łatwe
dodawanie nowych testów, a co za tym idzie, jest gotowa do rozbudowy
i dalszego rozwoju. W przyszłości można do niej dodać nowe testy, by
zwiększyć ich różnorodność lub poszerzyć zakres badań o różne warunki ich
wykonywania.

\subsection{Najeźdźcy z kosmosu}
W tej grze użytkownik porusza w lewo lub prawo działem, które strzela w górę
określonym odstępie czasowym. Celem gry jest zestrzelenie wszystkich statków zanim skończy się czas.

\begin{figure}[h]
\centering
\includegraphics[width=6cm]{images/space_invaders}
\caption{Wizualizacja: Najeźdźcy z kosmosu!}
\label{fig:space_invaders}
\end{figure}

W wersji utrudnionej, użytkownik dostaje mniej czasu, przeciwnicy szybciej
się poruszają, a działo strzela wolniej. Czas testu oraz odstęp między
strzałami powodują, że zestrzelenie wszystkich statków jest niewykonalne.

\subsection{Rozbij jajko}
Zadaniem użytkownika jest rozbicie jajka poprzez stuknięcie w nie
odpowiednią ilość razy.

\begin{figure}[h]
\centering
\includegraphics[width=2.6cm]{images/egg_smasher}
\caption{Wizualizacja: Rozbij Jajko!}
\label{fig:egg_smasher}
\end{figure}

W przerywanym wariancie jest mniej czasu oraz potrzeba znacznie większej
ilości uderzeń na każdy etap pękania. Teoretycznie w wariancie trudniejszym
zadanie wciąż jest możliwe do wykonania, lecz jest to trudne do osiągnięcia.

\subsection{Powódź}
Plansza gry składa się z kwadratów, których kolor jest losowy spośród sześciu
barw. Gdy gracz kliknie jeden z kolorów, zamalowuje nim dotychczas zalany
fragment planszy i dołącza do powodzi te kwadraty, które
teraz mają ten sam kolor co zalana powierzchnia i się z nią stykają.


\begin{figure}[h]
\centering
\includegraphics[width=3.7cm]{images/flood_it}
\caption{Wizualizacja: Powódź!}
\label{fig:flood_it}
\end{figure}

Zadaniem użytkownika jest spowodowanie, aby cała plansza była jednego koloru.
W trudniejszym wariancie czas jest skrócony, a plansza znacznie zwiększona, co powoduje, że potrzeba więcej ruchów, by wykonać zadanie.

\subsection{Uratuj księżniczkę}
Gracz steruje rycerzem, który musi przeskakiwać między trzema torami, by uniknąć przeciwników na swojej ścieżce. Gdy trafi na jakiegoś, traci parę sekund ograniczonego czasu, tak więc jeśli wpadnie na zbyt wielu przeciwników, nie zdąży dotrzeć do zamku.

\begin{figure}[h]
\centering
\includegraphics[width=6cm]{images/save_princess}
\caption{Wizualizacja: Uratuj księżniczkę!}
\label{fig:save_princess}
\end{figure}

Utrudniona wersja posiada skrócony czas oraz zwiększoną prędkość poruszania się postaci, przez co trudniej uniknąć zderzenia.

\subsection{Zator}
Użytkownik musi odblokować drogę dla wyznaczonego klocka tak aby ten
mógł opuścić planszę. Prostokątne klocki można przesuwać jedynie wzdłuż
dłuższych krawędzi, tak więc tylko do góry i w dół lub w lewo i prawo.

\begin{figure}[h]
\centering
\includegraphics[width=5.33cm]{images/unlock}
\caption{Wizualizacja: Zator!}
\label{fig:unlock}
\end{figure}

W utrudnionej wersji, poza skróconym czasem, gracz otrzymuje do rozwiązania o wiele trudniejszą wersję gry. Zagadka posiada więcej elementów i więcej zależnych od siebie akcji przesunięcia.

\subsection{Lodowy Labirynt}
Celem zadania jest przejście przez labirynt. Utrudnieniem jest fakt, że gdy
zacznie się poruszać w którymś kierunku, zmienić go może dopiero po napotkaniu ściany. Należy planować swoje ruchy pamiętając, że poruszać trzeba się do końca linii.

\begin{figure}[h]
\centering
\includegraphics[width=5.35cm]{images/lab_ice}
\caption{Wizualizacja: Lodowy labirynt!}
\label{fig:lab_ice}
\end{figure}

Trudniejszy wariant, oprócz skróconego czasu, posiada inaczej umiejscowione przeszkody.
Prędkość gracza ulega zmniejszeniu, przez co nawet przy znajomości dokładnej ścieżki, użytkownik nie jest w stanie ukończyć zadania w wyznaczonym czasie.

\subsection{Dziurawy labirynt}
Tym razem gracz, poprzez przechylanie ekranu w odpowiednią stronę, prowadzi przez labirynt kulę. Utrudnieniem jest fakt, że w planszy są dziury, których należy unikać. Jeśli kula wpadnie w dziurę, gra zaczyna się od nowa z miejsca startowego.

\begin{figure}[h]
\centering
\includegraphics[width=6cm]{images/lab_hole}
\caption{Wizualizacja: Dziurawy labirynt!}
\label{fig:lab_hole}
\end{figure}

W trudniejszym wariancie gry, użytkownik dostaje mniejszą ilość czasu
oraz większą ilość przeszkód na drodze do wyjścia.  

\subsection{Super Pamięć}
Gra polega na znalezieniu par identycznych obrazków. Za każdym razem,
gdy odsłonięte zostaną dwa takie same, są one wykluczane z gry. Gra kończy się
w momencie odkrycia ostatniej pary obrazków. W przypadku błędnego wskazania, po chwili obrazki są ponownie zakrywane.

\begin{figure}[h]
\centering
\includegraphics[width=6cm]{images/memory}
\caption{Wizualizacja: Super Pamięć!}
\label{fig:memory}
\end{figure}

W utrudnionym wariancie czas na znalezienie par jest skrócony, a karty dłużej zostają odsłonięte.
Ta gra jest wybrana jako ostatnia ze względu na to, że wykorzystane będą w niej obrazki z poprzednich gier. Ma to na celu jednocześnie spowodować, aby gracz skupił się na zapamiętywaniu innych rzeczy niż poprzednie gry, a jednocześnie podawać mu obrazkowe skojarzenia z wcześniejszymi zadaniami. 

\chapter{Projekt aplikacji i wybrane technologie}
W tym rozdziale omówione zostanie, jak została zaprojektowana aplikacja.
Przedstawiony zostanie podstawowy interfejs użytkownika oraz użyte technologie, w tym użycie GameMaker: Studio do stworzenia interfejsu i oprogramowania zadań w języku skryptowym GML, jak również użycie zewnętrznego serwera do zapamiętywania danych o wypełnionych ankietach przy pomocy GameMaker Server.

\section{Projekt interfejsu użytkownika}
Interfejs użytkownika został wykonany na podstawie pokoi, między którymi
przemieszcza się badany podczas robienia testu. Wszystkie zadania, które
trzeba wykonać mają ten sam główny interfejs. Tak jak to przedstawiono
na poniższej ilustracji, każde zadanie ma wyznaczone miejsce, w którym
pojawia się jego tytuł oraz krótki opis. W opisie mieszczą się przede
wszystkim warunki zwycięstwa oraz wytłumaczenie sterowania.

Większość ekranu jest przeznaczona na teren gry. Szara strefa wyznacza
granice gry i to w niej zawsze pojawiają się wszystkie potrzebne obiekty.
Każda gra wczytuje się na swój sposób. Podczas wczytywania się każdego z pokoi,
konkretne zadanie ładuje odpowiednie zmienne, które są inne w zależności od tego, czy gra ma być przez użytkownika skończona, czy też ma mu zostać przerwana.

\begin{figure}[H]
\centering
\includegraphics[width=6cm]{images/interfejs}
\caption{Projekt interfejsu użytkownika}
\label{fig:obrazek k}
\end{figure}

Pod miejscem na grę jest dodatkowa przestrzeń na elementy sterujące. Tam znajdują się przyciski pozwalające m.in. na przejście do następnego zadania w przypadku zakończenia jednego (niezależnie od tego, czy zadanie zostało ukończone, czy też przerwane przez upłynięcie wyznaczonego czasu).

\section{GameMaker: Studio}
Środowisko programistyczne GameMaker: Studio przeznaczone jest do tworzenia
gier i aplikacji w technologii dwuwymiarowej. Korzysta z własnego
języka skryptowego GML. Jak już pisałem wcześniej, wybrałem to środowisko
ze względu na ułatwione pisanie podstawowych funkcji, dzięki
czemu można bardziej skupić się na tych skryptach, które są ważne i
tworzą logikę gry.

Dla każdego z zadań przeznaczony jest osobny pokój, w którym generują
się wszystkie elementy niezbędne do przeprowadzenia gry. Pomimo, że jest to
oprogramowanie pozwalające na bardzo dużą swobodę wstawiania elementów
za pomocą trybu ”złap i upuść”, większość pokoi w mojej pracy posiada
jedynie dwa lub trzy kontrolery, które dynamicznie uzupełniają resztę zawartości w zależności od tego, co aktualnie powinno dziać się z aplikacją.

\section{Gromadzenie danych - GameMaker Server}
Głównym celem tej aplikacji jest zbieranie danych w celu badania efektu
Zeigarnik. Wyniki każdego użytkownika są spisywane po stronie serwera. W
tym celu wykorzystywane jest rozszerzenie do GameMaker: Studio o nazwie
GameMaker Server, które pozwala na bezproblemowe założenie serwera i
obsługę plików INI po jego stronie. Każdy użytkownik dostaje swój plik,
w który automatycznie zapisują się wszystkie wyniki jego testów, dane o tym, które zadania zostały wykonane poprawnie, a przy których skończył się czas oraz adnotację do tego, które z zadań były ograniczone.

Poza tym, w pliku na serwerze zostają zapisane także informacje przekazane
w ankiecie, w której to użytkownik proszony jest o przypomnienie
sobie jak największej ilości tytułów zadań oraz o podanie ich krótkiego opisu.

GameMaker Server przechowuje dane w pliku INI na zasadzie klucz i wartość, które mogą być posegregowane w odpowiednie sekcje. Tak więc wyniki
są podzielone na dwie sekcje: Pierwszą, w której mieszczą się informacje wypełniane
na bieżąco przez aplikację, np. które zadania zostały utrudnione lub czy zadanie
zostało wykonane czy też skończył się czas oraz drugą, w której gromadzone są odpowiedzi z ankiety, gdzie użytkownik może wpisać dowolną liczbę odpowiedzi tytuł-opis, dopóki nie stwierdzi, że nic więcej sobie nie przypomina.

Prostota tych danych pozwala na łatwą analizę odpowiedzi. W przyszłości
można stworzyć skrypt analizujący dane wejściowe, gdy tylko zostają dostarczone.
Jedynym problemem jest to, że użytkownik wpisuje własne odpowiedzi, a co za tym idzie, powierzenie analizy tekstu skryptowi nie zawsze da adekwatne wyniki, ponieważ może on źle zinterpretować odpowiedź i przypisać ją do nieodpowiedniego zadania.

\section{Pliki INI i organizacja danych}
Plik INI to prosty format zapisu danych, który pozwala na ich przechowywanie. Postanowiłem wykorzystać go w mojej aplikacji ze względu
na łatwość stworzenia osobnego pliku dla każdego użytkownika. W pliku zapisuje
dane na zasadzie klucz-wartość tak, jak w przypadku zwykłych baz danych.

Ułatwia to dalszą interpretację dostarczanych danych ze względu na
trudność w interpretacji odpowiedzi graczy. Oprócz podstawowego podsumowania, czy konkretne zadanie zostało ukończone lub nie, najważniejszym aspektem
jest to, które zadania użytkownik był w stanie zapamiętać. 

\begin{lstlisting}[caption={Fragment pliku INI}]
[test_made]
worse = 00110101
space_invaders = true
egg_smasher = true
flood_it = false
save_princess = true
unlock_it = false
\end{lstlisting}

Powyższy przykład pokazuje, że podstawowo w pliku zapisany jest ciąg zerojedynkowy  określający, które zadanie według kolejności posiadały utrudniony
wariant, a które były normalne. Następnie aplikacja zapisuje wynik
każdego zadania, gdy tylko użytkownik je zakończy.

 Wpisując odpowiednio
nazwę pokoju, w którym aktualnie znajduje się użytkownik oraz wartość „prawda”, jeśli zadanie zostało przez niego ukończone w czasie lub „fałsz”, jeśli czas się skończył (tak jak w ukazanym poniżej fragmencie skryptu wykonywanego w przypadku, gdy czas się skończy). Jako, że ważne jest, jak zostało wykonane zadanie, do wszystkich gier użyto tych samych skryptów kończących rozgrywkę.

\begin{lstlisting}[caption={Fragment skryptu w przypadku końca czasu}]
room_name = room_get_name(room);
gms_ini_player_write("test_made", room_name, false);
\end{lstlisting}

\chapter{Szczegóły implementacji}
W następującym rozdziale zajmę się sposobem implementacji poszczególnych
pokoi. Głównie można podzielić je na trzy kategorie:

-Pokoje inicjujące grę, w których należy zalogować się na serwer w celu przekazania danych oraz odczytania ostrzeżenia na temat limitu czasowego poszczególnych zadań i informacji, iż muszą one zostać wykonane w całości w jednym ciągu. 

-Pokoje testów, z których każdy ma zaimplementowaną w sobie inną grę. Oprogramowanie jest tak przystosowane, żeby pokoje były od siebie niezależne, a wszystkie potrzebne wyniki zapisywały się bez konieczności dopisywania dodatkowych skryptów. Tym samym, dodanie nowego pokoju z grą wymaga jedynie wpisania go na listę startową oraz zaimplementowania samej gry w tym pokoju.
Pokoje końcowe, przeznaczone na ankietę, którą użytkownik wypełnia po wykonaniu ostatniego zadania.


\section{Inicjalizacja testu}
Pierwsze trzy pokoje służą do zalogowania się do serwera, dzięki czemu możliwy
będzie zapis informacji o teście w pliku zdalnym. W trzecim pokoju znajduje się informacja o limicie czasowym zadań oraz przycisk rozpoczynający test. W momencie kliknięcia na niego, program losuje połowę pokoi (zaokrągloną w dół) i zmienia ich poziom trudności tak, aby nie dało się ich ukończyć w wyznaczonym czasie. 

\begin{lstlisting}[caption={Fragment skryptu scr\_make\_times}]
for(i=0;i<floor(tasks/2);i++){
    checked = false;
    while(!checked){
        x = floor(random_range(0,(tasks - 0.1)))
        if(!global.worse[x]){
            global.worse[x] = true;
            checked = true;
        }
    }
}
\end{lstlisting}

Za wylosowanie pokoi odpowiedzialny jest skrypt o nazwie scr\_make\_times, którego fragment widzimy powyżej. Fragment ten pokazuje sposób, w który losowane są pokoje. Tablica worse jest globalną zmienną i posiada tyle miejsc, ile jest zadań. Na każdym miejscu, dla konkretnego zadania, ma domyślnie wpisaną wartość false, która oznacza że gra nie jest utrudniona.

Algorytm losuje pokój i zmienia wartość worse dla tego pokoju na true, które powoduje utrudnienie zadania.


\section{Pokoje z grami}
Domyślnie każdy pokój ładuje się w ten sam sposób i, tak jak było to zaprojektowane, posiada miejsce, w którym pokazuje się gra. Na górze jest miejsce na tytuł zadania oraz wyjaśnienie zasad. Każdy z pokoi, w którym są gry, ładuje się wraz z zestawem zmiennych potrzebnych do jej działania. Zależnie od wcześniej wylosowanej wartości worse dla danego pokoju, wczytywane są inne wartości potrzebnych zmiennych.

\subsection{Najeźdźcy z Kosmosu}
W tej grze gracz porusza wieżą obronną. Przeciwnicy poruszają się w określony
sposób, a sama wieża strzela automatycznie w określonych odstępach czasu. Najważniejszym aspektem gry jest sterowanie wieżą w lewo i w prawo.

\begin{lstlisting}[caption={Skrypt scr\_direction}]
if(point_direction(xstart,ystart,x,y) > 270 
   or point_direction(xstart,ystart,x,y) < 90){
    obj_tower.direction_s = 1;
}

if(point_direction(xstart,ystart,x,y) > 90 
   and point_direction(xstart,ystart,x,y) < 270){
    obj_tower.direction_s = 2;
}
\end{lstlisting}

Gdy użytkownik przesunie palcem po ekranie, gra zapisuje współrzędne
początku oraz końca ruchu. Potem porównuje te dwa punkty, sprawdzając
pod jakim kątem znajdują się one względem siebie. Następnie, zależnie od kąta przesunięcia palca, wieża może zmienić swój kierunek. Zawarty powyżej skrypt
ukazuje sprawdzenie kąta przesunięcia oraz wybór, w którą stronę powinna
przemieszczać się wieża.

\subsection{Rozbij jajko}
Rozbicie jajka polega na stuknięciu w nie odpowiednią ilość razy. Zawsze, gdy otrzyma odpowiednią ilość uderzeń, pęka coraz bardziej, zmieniając się wizualnie.

\begin{lstlisting}[caption={Fragment  kodu obiektu obj\_egg}]
if (current_hit == 0){
    if(image_index &lt; 4){
        image_index += 1;
        current_hit = hit_count;
    } else{
        image_index += 1;
        depth = -150
        scr_end_win();
    }
}
\end{lstlisting}

Powyższy kod ukazuje sprawdzenie warunku zwycięstwa przy każdym kolejnym
uderzeniu. Jeśli została spełniona ilość ciosów, należy dodać więcej pęknięć. Kiedy już użytkownik rozbije w końcu jajko, następuje ostatnia zmiana - na obraz stłuczonego jajka oraz ekran końca gry.

\subsection{Powódź}
Ta gra polega na używaniu sześciu przycisków w celu zmienienia koloru dotychczas
pochłoniętych kwadratów. Po pokolorowaniu kwadratów, które posiadają wartość ”flood\_flooded[i,n] == true”, program sprawdza wszystkie przyległe pola (nie po skosie) w poszukiwaniu takiego samego koloru, na jaki pomalowana została reszta. W przypadku kiedy znajdzie taki sam kolor, wchłania go zmieniając parametr flood\_flooded z false na true i rekurencyjnie sprawdza dla tego nowego kwadratu, czy on też nie znajduje się obok innych o tym samym kolorze, które nie zostały jeszcze pochłonięte. 


\begin{lstlisting}[caption={Fragment  skryptu scr\_retrue\_colors}]
//checking up
if(i-1 >= 0 && obj_flood.flood_color[i-1,n] == arg_color){
    if(obj_flood.flood_flooded[i-1,n] != true){     
        obj_flood.flood_flooded[i-1,n] = true;
        scr_retrue_colors(i-1,n,arg_color);
    }   
}
\end{lstlisting}

Powyższy fragment ukazuje sprawdzenie, czy nad aktualnie pochłoniętym kwadratem nie znajduje się inny o tym samym kolorze. Analogicznie wyglądają
sprawdzenia dla kwadratu pod spodem, jak i z prawej i z lewej strony. Oczywiście nie sprawdzamy dalszą rekurencją tych kwadratów, które zostały już pochłonięte.

\subsection{Uratuj księżniczkę}
W tej grze, gracz musi unikać spotkań z wrogiem, aby jak najszybciej dostać się do księżniczki. Aby stworzyć iluzję nieskończonego poruszania się do przodu, to nie bohater się porusza lecz jego przeciwnicy. W przypadku dojścia do lewego brzegu planszy, przeciwnik pojawia się na losowym torze na prawym brzegu, tak jak demonstruje to poniższy kod. 

\begin{lstlisting}[caption={Fragment kodu obiektu obj\_orc}]
if(x &lt; 120){
    y = choose(750, 1000, 1250);
    x = 960;
    obj_hero.win_count -= 1;
}
\end{lstlisting}
Jeśli bohater trafi na przeciwnika to przestaje na chwile biec aby go
pokonać. Jeśli trafi na za dużą liczbę przeciwników, skończy mu się czas i
nie zdąży dobiec do księżniczki.

\subsection{Zator}
Zator polega na odsunięciu z drogi wszystkich klocków, które zasłaniają wyjście zielonemu blokowi. Problemem jest fakt, że wszystkie klocki można przesuwać tylko wzdłuż ich dłuższej krawędzi. Zważając na założenia tej gry, najważniejszym jej aspektem jest sprawdzanie kolizji pomiędzy klockami i zapobieżenie ewentualnemu nakładaniu się na siebie elementów. 

\begin{lstlisting}[caption={Fragment kodu obiektu obj\_green\_one}]
if(!(place_meeting(mouse_x + xx, y, obj_vertical_one)
    or place_meeting(mouse_x + xx, y, obj_horizontal_one)) 
    and (mouse_x+xx &lt;850 and mouse_x+xx &gt; 230)){
        if(grab == false){
            exit;
        } else{
            x = mouse_x + xx;
        }    
}
\end{lstlisting}

Jak widać na załączonym powyżej fragmencie kodu obiektu zielonego bloku, w
przypadku poruszenia jakimkolwiek blokiem, program wpierw sprawdza, czy wykonany ruch nie spowoduje kolizji i dopiero po upewnieniu się, że nic nie stoi na drodze, przesuwa blok tak, jak było to zaplanowane.


\subsection{Lodowy Labirynt}
Założenie Lodowego Labiryntu jest takie, że gdy zaczniemy się poruszać, nie możemy zmienić kierunku dopóki nie dotrzemy do ściany, która nas zatrzyma. Tym samym, aplikacja została zaprogramowana, by analizować ruch palca przy przesunięciu i na jego podstawie wyznaczyć kierunek ruchu. W momencie rozpoczęcia ruchu blokujemy skrypt odpowiedzialny za wybieranie kierunku.

\begin{lstlisting}[caption={Fragment kodu obiektu obj\_player\_ice}]
    if(place_meeting(x-speed_ice, y, obj_lab_wall)){
        while(!place_meeting(x-1, y, obj_lab_wall)){
            x -= 1;
        }
\end{lstlisting}
W przytoczonym fragmencie kodu widzimy, jak co klatkę sprawdzane jest położenie obiektu i czy nie styka się on już ze ścianą. Jest to fragment sprawdzający kolizję w przypadku ruchu w lewo. Dla każdego kierunku kod ten różni się ze względu na inne położenie na osiach.


\subsection{Dziurawy labirynt}
W przeciwieństwie do innych zadań, w tym nie sterujemy za pomocą ruchu palca, lecz poprzez przechylanie urządzenia. Program na bieżąco bada, czy i która oś jest przekręcona i dostosowuje do tego ruch kuli. W przypadku wpadnięcia w dziurę, poziom się resetuje.

\begin{lstlisting}[caption={Fragment kodu obiektu obj\_player\_ball}]
if(place_meeting(x+xx, y, obj_lab_wall)){
    while(!place_meeting(x-device_get_tilt_x(), y, obj_lab_wall)){
        x += -device_get_tilt_x();
    }
}else{
    x += xx;
}
\end{lstlisting}

Takie zachowanie wymagało innego algorytmu kolizji, ponieważ ważne było, aby kula przesuwała się po krawędzi w przypadku przechylenia urządzania w jej stronę. Powyższy przykład ukazuje tę funkcję dla osi X.


\subsection{Super Pamięć}
W tej grze, która jest ostatnią w całym teście, wykorzystane zostały grafiki z poprzednich zadań. Gdy gracz odkrywa pierwszą kartę, zapamiętane zostaje, jaki obrazek się pod nią krył. Gdy wybiera drugą kartę, program sprawdza zgodność przypisanego jej obrazu z tym z poprzedniej karty. Po tym sprawdzeniu decyduje się na jedną z dwóch akcji.

\begin{lstlisting}[caption={Fragment kodu obiektu obj\_memory\_card}]
    if(obj_memory_shuffle.now_turned == 1){
        obj_memory_shuffle.now_shown = obj_memory_shuffle.card_deck[| id_fac];
        obj_memory_shuffle.now_shown_id = id;
    } else if(obj_memory_shuffle.now_turned == 2){
        if(obj_memory_shuffle.now_shown == obj_memory_shuffle.card_deck[| id_fac]){
            alarm[0] = obj_memory_shuffle.delay_time;
        }else{
            alarm[1] = obj_memory_shuffle.delay_time;
        }
    }
\end{lstlisting}

W przypadku kiedy rysunki się ze sobą zgadzają, wywoływany zostaje alarm, który powoduje usunięcie obu instancji odkrytych kart. W drugim przypadku, gdy karty się ze sobą nie zgadzają, wywoływany jest alarm powodujący zakrycie kart z powrotem. Niezależnie od tego, który alarm zostanie wywołany, informacje o ostatnio odkrytej karcie zostają zresetowane.


\section{Ankieta}
Na potrzeby zakończenia badania, została zaimplementowana prosta ankieta, w której użytkownik ma za zadanie wpisać tytuły oraz krótkie opisy zadań, które wykonywał. Ankieta składa się na dwa pola tekstowe oraz przyciski umożliwiające wpisanie kolejnego zadania lub zakończenie wpisywania zadań. Tutaj nie ma żadnego limitu czasowego, użytkownik ma opisać to, co pamięta i tak, jak to pamięta. Wszystkie te dane oraz informacje o tym, które zadania użytkownik zdołał ukończyć zapisują sie na serwerze, dla każdego użytkownika w osobnym pliku.

% zakończenie
\chapter{Podsumowanie wyników testu}

Aplikacja została przedstawiona grupie dwudziestu sześciu osób. Przy analizie
danych brał udział Bartosz Tomkiewicz, magister psychologii. Z dostarczonych
nam przez aplikację danych dowiedzieliśmy się, że stosunek zapamiętanych przez użytkowników zadań nieukończonych do ukończonych wynosi 1,78. Oznacza to, że
osoby wypełniające test były w stanie przypomnieć sobie zadania przerwane
o 78\% częściej niż te, które udało im się dokończyć.

Taki wynik potwierdza Efekt Zeigarnik oraz to, że nasza pamięć często
pozbywa się informacji o czynnościach, które wykonaliśmy do końca, aby nie
zaprzątać nimi naszych myśli. Należy jednak pamiętać, że wynik ten to
średnia wszystkich użytkowników. W grupie badanych znalazła się zarówno osoba,
która pamiętała jedynie nieskończone zadania, jak i taka, która nie
przypomniała sobie żadnego z przerwanych testów.

Co ciekawe, oryginalne badania Blumy Zeigarnik przeprowadzone na podobnej ilościowo grupie ludzi, dały zbiorczy wynik 1,9 czyli zbliżony do uzyskanego za pomocą mojej aplikacji. W przyszłości można powtórzyć badania dodając do aplikacji więcej testów lub zmieniając wymagania obecnych oraz zwiększając liczbę użytkowników, aby uzyskać więcej wiarygodnych wyników.



% załączniki (opcjonalnie):
%\appendix
%\chapter{Tytuł załącznika jeden}

%Treść załącznika jeden.

%\chapter{Tytuł załącznika dwa}

%Treść załącznika dwa.

% literatura (obowiązkowo):
\bibliographystyle{unsrt}
\bibliography{xml}
\begin{thebibliography}{9}

\bibitem{Zeigarnik}
Bluma Zeigarnik, „On Finished and Unfinished Tasks”\\
\textit{\url{http://codeblab.com/wp-content/uploads/2009/12/On-Finished-and-Unfinished-Tasks.pdf/}}

\bibitem{GMS}
GameMaker: Studio Dokumentacja\\
\textit{\url{https://docs.yoyogames.com//}}(Dostęp: 25.05.2017)

\bibitem{GMServer}
GameMaker Server Dokumentacja\\
\textit{\url{http://gamemakerserver.com/en/docs//}}(Dostęp: 25.05.2017)

\bibitem{draw_text_shadow}
Użytkownik SP1D3R z forum YoYo Games, skrypt do cieni pod tekstem.\\
\textit{\url{https://forum.yoyogames.com/index.php?threads/text-shadow.23435/}}(Dostęp: 10.04.2017)

\bibitem{Knight Sprite}
Użytkownik bellow ze strony OpenGameArt.Org, grafika i animacja rycerza.\\
\textit{\url{https://opengameart.org/content/pixel-character-0}}(Dostęp: 10.04.2017)

\bibitem{Orc Sprite}
Użytkownik Calciumtrice ze strony OpenGameArt.Org, grafika i animacja orków.\\
\textit{\url{https://opengameart.org/content/animated-orcs}}(Dostęp: 10.04.2017)

\end{thebibliography}

% spis rysunków (jeżeli jest potrzebny):
\listoffigures

\oswiadczenie

\end{document}