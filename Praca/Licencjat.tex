%\documentclass[brudnopis]{xmgr}
% Jeśli nowe rozdziały mają się zaczynać na stronach nieparzystych:
\documentclass[openright]{xmgr}

% install minted package to highlight source code
 %\usepackage{minted}

%\defaultfontfeatures{Scale=MatchLowercase}
%\setmainfont[Numbers=OldStyle,Ligatures=TeX]{Minion Pro}
%\setsansfont[Numbers=OldStyle,Ligatures=TeX]{Myriad Pro}
% for fontspec version < 2.0
% \setmainfont[Numbers=OldStyle,Mapping=tex-text]{Minion Pro}
% \setsansfont[Numbers=OldStyle,Mapping=tex-text]{Myriad Pro}
%\setmonofont[Scale=0.75]{Monaco}

% Opcjonalnie identyfikator dokumentu
% drukowany tylko z włączoną opcją 'brudnopis':
\wersja   {wersja wstępna [\ymdtoday]}

\author   {Marcin Szpaderski}
\nralbumu {195\,008}
\email    {szpaderski.marcin@gmail.coml}


\title    {Aplikacja wspierająca badania nad efektem Zeigarnik}
\date     {2017}
\miejsce  {Gdańsk}

\opiekun  {dr Włodzimierz Bzyl}

% dodatkowe polecenia
%\renewcommand{\filename}[1]{\texttt{#1}}
%\definecolor{stress}{cmyk}{0,1,0.13,0} % RubineRed
%\definecolor{topic}{cmyk}{0.98,0.13,0,0.43} % MidnightBlue

\begin{document}

% streszczenie
\begin{abstract}
W pracy przedstawiono aplikację, która ma na celu wspieranie badań nad Efektem Zeigarnik. Aplikacja posiada serię prostych testów, które użytkownik ma za zadanie rozwiązać w określonym czasie. W celu badania Efektu Zeigarnik, czas dany na rozwiązanie zadań jest różny. Z założenia, przed rozpoczęciem testu aplikacja losuje połowę zadań i skraca ich czas tak aby nie dało się ich wykonać, lecz pozostawia wystarczająco czasu, aby zapoznać się z poleceniem. Potem użytkownik wypełnia ankietę, która pozwala ustalić, które z wykonanych zadań użytkownik jest w stanie sobie przypomnieć.

Do stworzenia aplikacji wykorzystane zostało środowisko do projektowania gier i programów komputerowych "GameMaker: Studio".  Wybrałem to oprogramowanie ze względu na chęć poszerzenia swojej wiedzy o umiejętność programowania w języku skryptowym GML. 

Wszystkie grafiki potrzebne do funkcjonowania aplikacji zostały wykonane przeze mnie.

Wszystkie podstawowe założenia aplikacji zostały wykonane. W przyszłości zostaną zaimplementowane nowe zadania.


\end{abstract}

% słowa kluczowe
\keywords{GameMaker Studio,
 Zeigarnik,
 Aplikacja,
 Psychologia,
 pamięć,
GML}

% tytuł i spis treści
\maketitle

% wstęp
\introduction

Zamiar niekoniecznie oznacza z góry określoną okazję do jego zrealizowania, lecz potrzebę lub tymczasową potrzebę, która stwarza taką okazję. Bulma Zeigarnik długo zastanawiała się nad tym stwierdzeniem, próbując zbadać jak wywołać u człowieka tę chwilową potrzebę, która wpływa na naszą pamięć. Celem niniejszej pracy jest stworzenie aplikacji, która będzie pełnić funkcję pomocniczą przy przeprowadzaniu badań nad tym, co dziś nazywamy Efektem Zeigarnik.

Efekt Zeigarnik został nazwany i opisany przez Blumę W. Zeigarnik\footnote{Bluma W. Zeigarnik (09.11.1900 - 24.02.1988) - rosyjska psycholog i psychiatra.} w 1927 roku. Opisuje on pojęcie psychologiczne związane z zagadnieniami pamięci i motywacji psychologii ogólnej. Efekt ten wykazuje, że czynności, które zostały nam przerwane jesteśmy w stanie lepiej sobie przypomnieć po pewnym czasie niż te, które wykonaliśmy bez żadnych problemów. Przykładem Efektu Zeigarnik są kelnerki w restauracjach, które jednocześnie pamiętają zamówienia nawet paru obsługiwanych w danym momencie osób, lecz gorzej przypominają sobie zamówienia klientów, którzy opuścili już lokal. Pomysł na aplikację, która pomaga badać ten efekt pojawił się podczas rozmowy ze znajomym, który ukończył studia na kierunku psychologia.

W części teoretycznej zostaną opisane główne założenia przyjęte podczas projektowania aplikacji. Przedstawione będą sposoby rozwiązania konkretnych problemów związanych z założeniami oraz możliwości aplikacji w zakresie przetwarzania informacji dostarczanych przez użytkowników.

Projekt będzie wykonany na zasadzie aplikacji działającej w trybie klient-serwer. Aplikacja zostanie zaprojektowana i wykonana w środowisku The GameMaker: Studio, które wykorzystuje unikalny dla siebie język GML, składnią zbliżony do C++ lub Pascal. Wybrałem takie środowisko ze względu na chęć rozszerzenia swojej wiedzy w zakresie implementacji programów i gier w środowiskach do tego przystosowanych. Wymieniona technologia zostanie opisana w niniejszej pracy, przedstawione będą zalety jej wykorzystania oraz szczegóły implementacji.




\chapter{Założenia aplikacji}
Głównym celem aplikacji jest stworzenie bazy informacji w celu przeprowadzenia badań nad Efektem Zeigarnik. Test jest jednorazowy, od użytkownika wymaga się, aby podszedł do niego jednorazowo oraz bez wiedzy, co tak naprawdę wykonuje.

Założeniem badania jest sprawdzenie, czy uczestnik po wykonaniu wszystkich zadań będzie w stanie łatwiej przypomnieć sobie te, które wykonał do końca, czy te, które zostały mu przerwane przed końcem.

Po wykonaniu wszystkich testów użytkownik dostanie do wypełnienia ankieta, która pozwoli ustalić, co zapamiętał badany. Wyniki poszczególnych zadań, jak i treść ankiety wysłane zostaną na serwer.

\section{Zadania}
Na pierwszą wersję aplikacji zaplanowane jest dziewięć różnych zadań. W momencie rozpoczęcia testu wylosowana zostanie połowa testów zaokrąglona w dół, a czas dany do ich rozwiązania zostanie skrócony tak, by użytkownik zdążył zaznajomić się z treścią zadania lub nawet zdążył zacząć je wykonywać, lecz aby na pewno nie zdążył go wykonać do końca.

Aplikacja jest przygotowana w taki sposób, że umożliwia stosunkowo łatwe dodawanie nowych testów, a co za tym idzie, jest gotowa do rozbudowy i dalszego rozwoju. W przyszłości można do niej dodać nowe testy, by zwiększyć ich różnorodność lub poszerzyć zakres badań o różne warunki ich wykonywania.
 
\subsection{Anagram}
W tym zadaniu użytkownik dostanie zbiór losowo poukładanych liter z których będzie musiał ułożyć prawidłowe słowo.

[Screenshot zadania][Do wstawienia]

Każdą z podanych liter można wykorzystać tylko i wyłącznie raz. Można natomiast cofnąć wybór liter i ustawić je w innej kolejności.

\subsection{Narysuj}
Zadaniem użytkownika jest narysowanie wazonu z kwiatami. Rysuje poprzez ruch palcem.

[Screenshot zadania][Do wstawienia]

Nie ma możliwości edycji tego, co zostało już narysowane. Usunięcie wyrysowanego kształtu jest niemożliwe.

\subsection{Zapałki}
Przedstawiona zostaje zagadka z zapałkami, a użytkownik ma za zadanie przesunąć jedną zapałkę tak, aby równanie było prawidłowe.

[Screenshot zadania][Do wstawienia]

Zapałek nie można odwracać. Można przesunąć jedynie jedną zapałkę. Po przesunięciu można zaakceptować odpowiedź albo wrócić wszystko na miejsce i wybrać inną zapałkę do przesunięcia.

\subsection{Matematyka}
Użytkownik dostaje do obliczenia stosunkowo łatwe równanie matematyczne. Należy podać prawidłowy wynik i go zaakceptować.

[Screenshot zadania][Do wstawienia]


\subsection{Ile to jest?}
Jest to nic innego jak równanie ze zmiennymi. Za pomocą trzech prostych równań użytkownik musi wyznaczyć równowartość niewiadomych.

[Screenshot zadania][Do wstawienia]

Po ustaleniu wyników użytkownik musi zaakceptować swoją ostateczną odpowiedź.

\subsection{Znajdź słowo}
Wśród wielu różnych liter użytkownik musi znaleźć konkretny wyraz i go zakreślić. 

[Screenshot zadania][Do wstawienia]

Dopóki nie zaakceptuje się odpowiedzi można dowolną ilość razy anulować zakreślenie i wybrać coś innego.

\subsection{Zakręć}
W tym zadaniu użytkownik ma przed sobą serię połączonych ze sobą zębatek, a jego celem jest określenie, czy ostatnia zębatka zakręci się w lewą czy w prawą stronę. Od początku znamy tylko kierunek obracania się pierwszej zębatki.

[Screenshot zadania][Do wstawienia]

Odpowiedź jest jednoznaczna, więc po jej zaakceptowaniu nie ma możliwości zmiany zdania.

\subsection{Labirynt}
Użytkownik ma za zadanie przejść przez labirynt. Utrudnieniem jest to, że gdy zacznie ruch w jakimkolwiek kierunku będzie się w nim poruszać dopóki nie natrafi na ścianę.

[Screenshot zadania][Do wstawienia]

Labirynt można zresetować do początkowego ustawienia w każdym momencie, jeśli gracz uzna że utknął.

\subsection{Puzzle}
Użytkownikowi zostaje dany do ułożenia obrazek bez jednego fragmentu. Jedynym sposobem na ułożenie go jest przesuwanie przyległych części na puste miejsce tak, by ułożyć z powrotem pełen obraz.

[Screenshot zadania][Do wstawienia]




\chapter{Projekt aplikacji i analiza potrzeb}
 
\section{Projekt interfejsu użytkownika}
\section{Wymagania funkcjonalne}
\section{Wymagania niefunkcjonalne}
\section{Organiacja danych}

\chapter{Opis wybranych technologii i rozwiązań}

\section{GameMaker Studio}

\chapter{Szczegóły implementacji}
 
\section{Serwer danych}
\section{Obsługa klienta}



% zakończenie
\summary

PODSUMOWANIE TUTAJ

% załączniki (opcjonalnie):
\appendix
\chapter{Tytuł załącznika jeden}

Treść załącznika jeden.

\chapter{Tytuł załącznika dwa}

Treść załącznika dwa.

% literatura (obowiązkowo):
\bibliographystyle{unsrt}
\bibliography{xml}

% spis tabel (jeżeli jest potrzebny):
\listoftables

% spis rysunków (jeżeli jest potrzebny):
\listoffigures

\oswiadczenie

\end{document}
