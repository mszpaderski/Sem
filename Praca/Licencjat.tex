%\documentclass[brudnopis]{xmgr}
% Jeśli nowe rozdziały mają się zaczynać na stronach nieparzystych:
\documentclass[openright]{xmgr}

% install minted package to highlight source code
%\usepackage{minted}
\usepackage{listings}
\usepackage{float}

% Define Colors
\usepackage{color}
\definecolor{eclipseBlue}{RGB}{42,0.0,255}
\definecolor{eclipseGreen}{RGB}{63,127,95}
\definecolor{eclipsePurple}{RGB}{127,0,85}
\definecolor{eclipseGreem}{RGB}{0,66,0}
\definecolor{eclipseRed}{RGB}{127,20,20}

% Define Language
\lstdefinelanguage{GML}
{
  % list of keywords
  morekeywords=[1]{
    if, else,
    while, exit,
    for
  },
morekeywords=[2]{room_name, room, checked, tasks, i, worse, x, point_direction, direction_s, y, current_hit, hit_count, flood_color, arg_color, n, flood_flooded, win_count, grab, xx, yy, speed_ice, now_turned, now_shown, card_deck, id_fac, delay_time, now_shown_id},
morekeywords=[3]{global, true, false, room_get_name, gms_ini_player_write, floor, random_range, xstart, ystart, image_index, depth, choose, place_meeting, mouse_x, mouse_y, device_get_tilt_x, id, alarm,  },
  sensitive=false, % keywords are not case-sensitive
  morecomment=[l]{//}, % l is for line comment
  morecomment=[s]{/*}{*/}, % s is for start and end delimiter
  morestring=[b]" % defines that strings are enclosed in double quotes
}
 
% Set Language
\lstset{
  language={GML},
  basicstyle=\small\ttfamily, % Global Code Style
  captionpos=b, % Position of the Caption (t for top, b for bottom)
  extendedchars=true, % Allows 256 instead of 128 ASCII characters
  tabsize=2, % number of spaces indented when discovering a tab 
  columns=fixed, % make all characters equal width
  keepspaces=true, % does not ignore spaces to fit width, convert tabs to spaces
  showstringspaces=false, % lets spaces in strings appear as real spaces
  breaklines=true, % wrap lines if they don't fit
  frame=trbl, % draw a frame at the top, right, left and bottom of the listing
  frameround=tttt, % make the frame round at all four corners
  framesep=4pt, % quarter circle size of the round corners
  numbers=left, % show line numbers at the left
  numberstyle=\tiny\ttfamily, % style of the line numbers
  commentstyle=\color{eclipseGreen}, % style of comments
  keywordstyle=[1]\color{eclipsePurple}, % style of keywords
  keywordstyle=[2]\color{eclipseGreem}, % style of keywords
  keywordstyle=[3]\color{eclipseRed}, % style of keywords
  stringstyle=\color{eclipseBlue}, % style of strings
}



%\defaultfontfeatures{Scale=MatchLowercase}
%\setmainfont[Numbers=OldStyle,Ligatures=TeX]{Minion Pro}
%\setsansfont[Numbers=OldStyle,Ligatures=TeX]{Myriad Pro}
% for fontspec version < 2.0
% \setmainfont[Numbers=OldStyle,Mapping=tex-text]{Minion Pro}
% \setsansfont[Numbers=OldStyle,Mapping=tex-text]{Myriad Pro}
%\setmonofont[Scale=0.75]{Monaco}

% Opcjonalnie identyfikator dokumentu
% drukowany tylko z włączoną opcją 'brudnopis':
\wersja   {wersja wstępna [\ymdtoday]}

\author   {Marcin Szpaderski}
\nralbumu {195\,008}
\email    {szpaderski.marcin@gmail.coml}


\title    {Aplikacja wspierająca badania nad efektem Zeigarnik}
\date     {2017}
\miejsce  {Gdańsk}

\opiekun  {dr Włodzimierz Bzyl}

% dodatkowe polecenia
%\renewcommand{\filename}[1]{\texttt{#1}}
%\definecolor{stress}{cmyk}{0,1,0.13,0} % RubineRed
%\definecolor{topic}{cmyk}{0.98,0.13,0,0.43} % MidnightBlue

\begin{document}

% streszczenie
\begin{abstract}
W ramach pracy napisano aplikację na urządzenia mobilne o nazwie "Pułapka Zeigarnik". Do stworzenia aplikacji wykorzystane zostało środowisko do projektowania gier i programów komputerowych ”GameMaker: Studio”. Napisano osiem testów, które użytkownik ma za zadanie rozwiązać w określonym czasie.

Wszystkie grafiki potrzebne do funkcjonowania aplikacji zostały wykonane przeze mnie w programie Paint.NET .

Aplikacja została wdrożona i umieszczona w repozytorium GitHub. Jest dostępna do pobrania ze strony: \url{ https://github.com/mszpaderski/Sem/tree/master/Pu\%C5\%82apka\%20Zeigarnik} .

Dla ułatwienia dostępu do aplikacji przygotowałem wersję testową, możliwą do zainstalowania na urządzeniu z systemem Android jak i Windows. Wersja na system Windows może nie działać prawidłowo ze względu na nieprzystosowane sterowanie. W pliku README.md znajdują się wymagane do instalacji informację. \url{ https://github.com/mszpaderski/Sem/tree/master/Pu\%C5\%82apka\%20Zeigarnik/Test_Version} .




\end{abstract}

% słowa kluczowe
\keywords{pamięć ludzka, efekt Zeigarnik, aplikacje mobilne, środowisko programistyczne GameMaker, GameMaker Server}

% tytuł i spis treści
\maketitle

% wstęp
\introduction

Zamiar niekoniecznie oznacza z góry określoną okazję do jego zrealizowania, lecz potrzebę lub tymczasową potrzebę, która stwarza taką okazję. Bulma Zeigarnik długo zastanawiała się nad tym stwierdzeniem, próbując zbadać jak wywołać u człowieka tę chwilową potrzebę, która wpływa na naszą pamięć. Celem niniejszej pracy jest stworzenie aplikacji, która będzie pełnić funkcję pomocniczą przy przeprowadzaniu badań nad tym, co dziś nazywamy Efektem Zeigarnik.

Efekt Zeigarnik został nazwany i opisany przez Blumę W. Zeigarnik\footnote{Bluma W. Zeigarnik (09.11.1900 - 24.02.1988) - rosyjska psycholog i psychiatra.} w 1927 roku. Opisuje on pojęcie psychologiczne związane z zagadnieniami pamięci i motywacji psychologii ogólnej. Efekt ten wykazuje, że czynności, które zostały nam przerwane jesteśmy w stanie lepiej sobie przypomnieć po pewnym czasie niż te, które wykonaliśmy bez żadnych problemów. Przykładem Efektu Zeigarnik są kelnerki w restauracjach, które jednocześnie pamiętają zamówienia nawet paru obsługiwanych w danym momencie osób, lecz gorzej przypominają sobie zamówienia klientów, którzy opuścili już lokal. Pomysł na aplikację, która pomaga badać ten efekt pojawił się podczas rozmowy ze znajomym Bartoszem Tomkiewiczem, który ukończył studia magisterskie na kierunku psychologia i który postanowił pomóc mi z przygotowaniem zadań oraz testów które będą nadawać się do badania efektu Zeigarnik. Analiza wyników też jest robiona z jego pomocą.

W części teoretycznej zostaną opisane główne założenia przyjęte podczas projektowania aplikacji. Przedstawione będą sposoby rozwiązania konkretnych problemów związanych z założeniami oraz możliwości aplikacji w zakresie przetwarzania informacji dostarczanych przez użytkowników.

Projekt będzie wykonany na zasadzie aplikacji mobilnej. Aplikacja zostanie zaprojektowana i wykonana w środowisku The GameMaker: Studio, które wykorzystuje unikalny dla siebie język GML, składnią zbliżony do C++ lub Pascal. Wybrałem takie środowisko ze względu na chęć rozszerzenia swojej wiedzy w zakresie implementacji programów i gier w środowiskach do tego przystosowanych. Wymieniona technologia zostanie opisana w niniejszej pracy, przedstawione będą zalety jej wykorzystania oraz szczegóły implementacji.




\chapter{Założenia aplikacji}
Głównym celem aplikacji jest stworzenie bazy informacji w celu przeprowadzenia badań nad Efektem Zeigarnik. Test jest jednorazowy, od użytkownika wymaga się, aby podszedł do niego tylko raz oraz bez wiedzy, co tak naprawdę wykonuje.

Założeniem badania jest sprawdzenie, czy uczestnik po wykonaniu wszystkich zadań będzie w stanie łatwiej przypomnieć sobie te, które wykonał do końca, czy te, które zostały mu przerwane przed końcem.

Po wykonaniu wszystkich testów użytkownik dostanie do wypełnienia ankieta, która pozwoli ustalić, co zapamiętał badany. Wyniki poszczególnych zadań, jak i treść ankiety wysłane zostaną na serwer.

\section{Porównanie wybranych środowisk programistycznych}
Aplikacja skierowana jest na tablety i inne urządzenia przenośne. Została stworzona w środowisku programistycznym o nazwie GameMaker: Studio. Środowisko to wybrałem spośród wielu innych dostępnych na rynku. W tej chwili najpopularniejszymi środowiskami wydają się być Microsoft Visual Studio oraz Unity. W tym rozdziale wyznaczę jakie są różnice pomiędzy nimi a tym które ostatecznie wybrałem oraz postaram się wyjaśnić dlaczego wybrałem to środowisko.

\subsection{Unity a GameMaker}
Środowisko Unity jest bardzo podobne do wybranego przeze mnie środowiska GameMaker. Służy głównie jako środowisko do tworzenia trójwymiarowych lub dwuwymiarowych gier komputerowych lub materiałów interaktywnych takich jak wizualizacje czy animację. Tak samo jak GameMaker, służy do tworzenia aplikacji na przeglądarki internetowe, komputery osobiste, konsole gier wideo oraz wszelkie urządzenia mobilne. Oba środowiska posiadają unikalne języki skryptowe. Unity wykorzystuje UnityScript, a GameMaker swój GML. Oba te języki są bazowane na języku JavaScript, i nie różnią się wiele składnią od niego. 

Główną różnicą pomiędzy tymi dwoma środowiskami, która przekonała mnie do GameMakera, do fakt że Unity jest narzędziem głównie do pracy nad elementami 3D, gdy mi zależało na aplikacji w dwóch wymiarach. Nie jest niemożliwe stworzenie zwykłej aplikacji, lecz w mojej opinii lepiej jest korzystać z narzędzi nastawionych przede wszystkim na żądany efekt.

\subsection{Microsoft Visual Studio a GameMaker}
Środowisko Microsoft Visual Studio pozwala na tworzenie aplikacji na wiele systemów, jednak jest skupiony na tworzeniu aplikacji dla systemów wspierających Windowsa. Co za tym idzie, aby zrobić aplikację nastawioną na urządzenia mobilne z systemem Android, niestety trzeba dodatkowych frameworków oraz ograniczeń których nie byłoby w przypadku programowania pod Windows Phone. Tym samym programując w GameMakerze istnieje możliwość łatwej zmiany aplikacji tak aby działała pod oboma systemami bez zbędnych formalności.

Podobnie jest z językiem programowania. Pomimo że teoretycznie można w Visual Studio pisać w prawie dowolnym języku programowania, to domyślnym i najbardziej dopracowanym językiem w Visual Studio jest C\# w przeciwieństwie do pozostałych dwóch środowisk które skupiają się na języku JavaScript. 



\section{Zadania}
Na pierwszą wersję aplikacji zaplanowane jest osiem różnych zadań. W momencie rozpoczęcia testu wylosowana zostanie połowa testów zaokrąglona w dół, a czas dany do ich rozwiązania zostanie skrócony tak, by użytkownik zdążył zaznajomić się z treścią zadania lub nawet zdążył zacząć je wykonywać, lecz aby na pewno nie zdążył go wykonać do końca.

Aplikacja jest przygotowana w taki sposób, że umożliwia stosunkowo łatwe dodawanie nowych testów, a co za tym idzie, jest gotowa do rozbudowy i dalszego rozwoju. W przyszłości można do niej dodać nowe testy, by zwiększyć ich różnorodność lub poszerzyć zakres badań o różne warunki ich wykonywania.
\subsection{Najeźdźcy z kosmosu}
W tej grze użytkownik porusza w lewo lub prawo, działo które strzela w określonym odstępie. Celem gry jest zestrzelenie wszystkich statków zanim skończy się czas.

\begin{figure}[h]
\centering
\includegraphics[width=6cm]{images/space_invaders}
\caption{Wizualizacja: Najeźdźcy z kosmosu!}
\label{fig:space_invaders}
\end{figure}

W wersji utrudnionej, użytkownik dostaje mniej czasu, przeciwnicy szybciej się poruszają a działo strzela wolniej. Czas testu oraz odstęp między strzałami powodują że nawet bezbłędny strzelec nie zdoła zestrzelić wszystkich statków.

\subsection{Rozbij jajko}
Zadaniem użytkownika jest rozbicie jajka. Zrobi to przez stuknięcie w nie odpowiednią ilość razy.

\begin{figure}[h]
\centering
\includegraphics[width=3cm]{images/egg_smasher}
\caption{Wizualizacja: Rozbij Jajko!}
\label{fig:egg_smasher}
\end{figure}

W przerywanym wariancie jest mniej czasu oraz potrzeba znacznie większej ilości uderzeń na każdy etap pękania. Teoretycznie w wariancie trudniejszym zadanie wciąż jest do wykonania, lecz jest to trudne do osiągnięcia.

\subsection{Powódź}
Plansza gry składa się z kwadratów których kolor jest losowy spośród sześciu barw. Gdy gracz kliknie jeden z kolorów to przemalowuje dotychczas zalany fragment planszy na dany kolor, i dołącza do powodzi te kwadraty które teraz mają ten sam kolor i stykają się z zalaną powierzchnią. Celem jest spowodowanie aby cała plansza była jednego koloru.

[Screenshot zadania][Do wstawienia]

W trudniejszym wariancie jest mniej czasu oraz znacznie zwiększona jest plansza, co powoduje że potrzeba więcej ruchów na wykonanie zadania.

\subsection{Uratuj księżniczkę}
Gracz musi przeskakiwać po trzech torach i unikać przeciwników na swojej ścieżce. Gdy trafi na jakiegoś, traci parę sekund ograniczonego czasu, tak więc jeśli wpadnie na za dużo przeciwników, nie zdąży dotrzeć do zamku.

[Screenshot zadania][Do wstawienia]

Utrudniona wersja posiada więcej przeciwników oraz szybciej się porusza, przez co trudniej uniknąć zderzenia. Poza tym czas jest krótszy.

\subsection{Zator}
Użytkownik musi odblokować drogę dla wyznaczonego klocka tak aby ten mógł opuścić planszę. Prostokątne klocki można przesuwać tylko wzdłuż dłuższych krawędzi, tak więc tylko do góry i w dół, albo w lewo i prawo. 

[Screenshot zadania][Do wstawienia]

Poza skróconym czasem, gracz otrzymuje do rozwiązania o wiele trudniejszą wersję gry. Zagadka posiada więcej elementów i więcej zależnych od siebie akcji przesunięcia.

\subsection{Lodowy Labirynt}
Celem zadania jest przejście przez labirynt. Utrudnieniem jest fakt że gdy zacznie poruszać się w którymś kierunku zatrzymać się i zmienić kierunek może dopiero po napotkaniu ściany. Trzeba planować swoje ruchy pamiętając że poruszyć trzeba się do końca linii.

[Screenshot zadania][Do wstawienia]

Trudniejszy wariant oprócz skróconego czasu, posiada większą ilość przeszkód. Prędkość gracza ulega zmniejszeniu, to oraz zmniejszony czas powodują że nawet przy znajomości dokładnej ścieżki, użytkownik nie jest w stanie ukończyć zadania w wyznaczonym czasie.

\subsection{Dziurawy labirynt}
Tym razem gracz prowadzi przez labirynt kulę, prowadzi ją poprzez przechylanie ekranu w odpowiednią stronę. Trudnością jest fakt że w planszy są dziury których należy unikać. Jeśli kula wpadnie w dół, gra zaczyna się od nowa.

[Screenshot zadania][Do wstawienia]

W trudniejszym wariancie gry, użytkownik dostaje mniejsza ilość czasu oraz większa ilość przeszkód na drodze do wyjścia.  

\subsection{Super Pamięć}
Gra polega na znalezieniu dwóch takich samych obrazków. Za każdym razem gdy odsłonięte są dwa takie same są one wykluczane z gry. Gra się kończy w momencie odkrycia ostatniej pary obrazków. W przypadku błędnego wskazania, po chwili obrazki są ponownie zakrywane.

[Screenshot zadania][Do wstawienia]

W utrudnionym wariancie jest mniej czasu na znalezienie wszystkich par. Poza tym jest o wiele więcej par do znalezienia.
Ta gra jest wybrana jako ostatnia ze względu na to że wykorzystane będą w niej obrazki z poprzednich gier. Ma to na celu jednocześnie spowodować aby gracz skupił się na zapamiętywaniu innych rzeczy niż poprzednie gry, i jednocześnie podawać mu obrazkowe skojarzenia z poprzednimi zadaniami.

\chapter{Projekt aplikacji i wybrane technologie}
W tym rozdziale przedstawione zostanie jak została zaprojektowana aplikacja. Omówiony zostanie podstawowy interfejs użytkownika oraz jakie zostały użyte technologie. W tym użycie GameMaker: Studio do stworzenia interfejsu oraz oprogramowania zadań w języku skryptowym GML czy użycie zewnętrznego serwera do zapamiętywania danych o wypełnionych ankietach dzięki użyciu GameMaker Server.  

\section{Projekt interfejsu użytkownika}
Interfejs użytkownika został wykonany na podstawie pokoi między którymi przemieszcza się badany podczas robienia testu. Wszystkie zadania które trzeba wykonać mają ten sam główny interfejs. Tak jak to przedstawiono na poniższej ilustracji, każde zadanie ma wyznaczone miejsce w którym pojawia się tytuł zadania oraz jego krótki opis. W opisie mieszczą się przede wszystkim warunki zwycięstwa oraz wytłumaczenie sterowania.

Większość ekranu jest przeznaczona na teren gry. Szara strefa wyznacza granice gry, i w tej strefie zawsze pojawiają się wszystkie potrzebne obiekty. Każda gra wczytuje się na swój sposób, podczas wczytywania każdego z pokoi konkretne zadanie ładuje odpowiednie zmienne które są inne w zależności czy gra ma być przez użytkownika skończona czy też ma być mu przerwana.

\begin{figure}[H]
\centering
\includegraphics[width=6cm]{images/interfejs}
\caption{Projekt interfejsu użytkownika}
\label{fig:obrazek k}
\end{figure}

Pod miejscem na grę znajduje się dodatkowa przestrzeń na elementy sterujące. Tam się znajdują przyciski pozwalające chociażby przejście do następnego zadania w przypadku ukończenia zadania, niezależnie od tego czy skończył mu się czas czy też zdołał przejść zadanie.

\section{GameMaker: Studio}
Środowisko programistyczne GameMaker: Studio przeznaczone jest do tworzenia gier i aplikacji w technologii dwuwymiarowej. Korzysta z własnego języka skryptowego GML. Tak jak już pisałem wcześniej, wybrałem to środowisko ze względu na ułatwienie w tworzeniu podstawowych funkcji, dzięki temu można lepiej skupić się na tworzeniu tych skryptów które są ważne i tworzą logikę gry.

Dla każdego z zadań jest przeznaczony osobny pokój w którym generują się wszystkie niezbędne do przeprowadzenia gry, rzeczy. Pomimo że jest to oprogramowanie które pozwala na bardzo dużą swobodę wstawiania elementów za pomocą trybu "złap i upuść" w mojej pracy większość pokoi posiada jedynie umieszczone w nich dwa lub trzy kontrolery które dynamicznie uzupełniają resztę zawartości w zależności od tego co się dzieje z aplikacją. 

\section{Gromadzenie danych - GameMaker Server}
Głównym celem tej aplikacji jest zbieranie danych w celu badania efektu Zeigarnik. Wyniki każdego użytkownika są spisywane po stronie serwera. W tym celu wykorzystywane jest rozszerzenie do GameMaker: Studio o nazwie GameMaker Server, który pozwala na bezproblemowe założenie serwera i obsługę plików INI po stronie serwera. Każdy użytkownik dostaje swój plik w który automatycznie wpisują się wszystkie wyniki testów, to czy został wykonany poprawnie czy skończył się czas, oraz adnotację do tego które z zadań były ograniczone.

Poza tym, w pliku na serwerze zostają zapisane także informację przekazywane z ankiety. W ankiecie użytkownik proszony jest o przypomnienie sobie jak największej ilości tytułów oraz podania krótkiego opisu co musiał wykonać w tych zadaniach. 

GameMaker Server przechowuje dane w pliku INI na zasadzie klucz i wartość oraz mogą być posegregowane w odpowiednie sekcję. Tak więc wyniki są podzielone na dwie sekcję, pierwszą w której mieszczą się informację wypełniane na bieżąco przez aplikację, takie jak które zadania zostały wybrane jako te które należy przerwać, oraz po każdym zadaniu informacja czy zadanie zostało wykonane czy też skończył się czas. Druga sekcja jest wydzielona na odpowiedzi w ankiecie, gdzie użytkownik może teoretycznie wpisać dowolną ilość odpowiedzi tytuł-opis dopóki nie stwierdzi że to wszystko co sobie przypomina.

Prostota tych danych pozwala na łatwą analizę odpowiedzi. W przyszłości można stworzyć skrypt analizujący dane wejściowe gdy tylko zostają dostarczone. Jedynym problemem jest to że użytkownik może wpisywać własne odpowiedzi, co za tym idzie, czasem trudno jest powierzyć analizę tekstu skryptowi który może źle zinterpretować o które zadanie chodziło użytkownikowi.

\section{Pliki INI i organizacja danych}
Plik INI jest to prosty format zapisu danych który pozwala na przechowywanie danych. Postanowiłem wykorzystać go w mojej aplikacji ze względu na prostotę stworzenia osobnego pliku dla każdego badanego. W pliku zapisuje dane na zasadzie klucz-wartość, jak w przypadku zwykłych baz danych. Ułatwia to jednak dalszą interpretację dostarczanych danych ze względu na trudność w interpretacji odpowiedzi graczy. Oprócz podstawowego podsumowania czy ukończono lub nie, konkretne zadanie, najważniejszym aspektem jest to które zadania użytkownik był w stanie zapamiętać.

\begin{lstlisting}[caption={Fragment pliku INI}]
[test_made]
worse = 00110101
space_invaders = true
egg_smasher = true
flood_it = false
\end{lstlisting}

Powyższy przykład obrazuje że podstawowo w pliku zapisany jest ciąg zero jedynkowy który określa które zadanie według kolejności posiadały utrudniony wariant, a które były normalne. Następnie aplikacja wpisuje wynik każdego zadania gdy tylko użytkownik je zakończy. Wpisując odpowiednio nazwę pokoju w którym aktualnie się znajdujemy oraz wartość prawdy, jeśli ukończyliśmy zadanie w czasie, lub fałsz, jeśli skończył nam się czas. Tak jak to przedstawia poniżej ukazany fragment skryptu wykonywanego w przypadku gdy czas się skończy. Jako że ważne jest jak zostało wykonane zadanie, do wszystkich gier zostały użyte te same skrypty kończące rozgrywkę.  

\begin{lstlisting}[caption={Fragment skryptu w przypadku końca czasu}]
room_name = room_get_name(room);
gms_ini_player_write("test_made", room_name, false);
\end{lstlisting}




\chapter{Szczegóły implementacji}
W następującym rozdziale zajmę się sposobem implementacji poszczególnych pokoi. Głównie można podzielić wszystkie pokoje na pokoje inicjujące grę, w których należy się zalogować na serwer w celu przekazywania danych i ostrzeżeniu że cały test jest robiony na czas więc trzeba znaleźć wystarczająco czasu aby wykonać go w całości w jednym ciągu. Drugą kategorią pokoi są te w których znajdują się testy, każdy z nich ma zaimplementowaną w sobie inną grę. Oprogramowanie jest tak przystosowane żeby pokoje były od siebie niezależne, a wszystkie potrzebne wyniki zapisywały się bez potrzeby dopisywania dodatkowych skryptów. Tym samym dodanie nowego pokoju z grą wymaga jedynie wpisania go na listę startową oraz zaimplementowanie samej gry w tym pokoju. Trzecim i ostatnim rodzajem pokoju są pokoje końcowe, przeznaczone na ankietę po wykonaniu ostatniego zadania.

\section{Inicjalizacja testu}
Pierwsze trzy pokoje służą do zalogowania się do serwera dzięki czemu możliwy będzie zapis informacji o teście w pliku zdalnym. W trzecim pokoju znajduje się informacja o tym że czas na wykonanie każdego zadania będzie ograniczony, oraz przycisk rozpoczynający test. W momencie kliknięcia na ten przycisk, program losuje połowę pokoi zaokrągloną w dół i zmienia ich poziom trudności tak aby nie dało się ich ukończyć w czasie.

\begin{lstlisting}[caption={Fragment skryptu scr\_make\_times}]
for(i=0;i<floor(tasks/2);i++){
    checked = false;
    while(!checked){
        x = floor(random_range(0,(tasks - 0.1)))
        if(!global.worse[x]){
            global.worse[x] = true;
            checked = true;
        }
    }
}
\end{lstlisting}

Za wylosowanie pokoi odpowiedzialny jest skrypt o nawie scr\_make\_times, którego fragment widzimy powyżej. Fragment ten pokazuje sposób w który wylosowywane są pokoje. Tablica worse jest globalną zmienną i posiada tyle miejsc ile jest zadań. Na każdym miejscu, dla konkretnego zadania, ma domyślnie wpisaną wartość false, która oznacza że gra nie jest utrudniona. Potem algorytm losuje pokój i zmienia wartość worse dla tego pokoju na true które powoduje utrudnienie zadania.

\section{Pokoje z grami}
Domyślnie każdy pokój ładuje się w ten sam sposób, tak jak było to zaprojektowane, posiada miejsce w którym pokazuje się gra. Na górze jest miejsce na tytuł zadania oraz wyjaśnienie zasad. Każdy z pokoi w którym są gry, ładuje się z zestawem zmiennych potrzebnych do działania gry. Zależnie od wcześniej wylosowanej wartości worse dla danego pokoju, wczytywane są inne wartości potrzebnych zmiennych.

\subsection{Najeźdźcy z Kosmosu}
W tej grze, gracz porusza wieżą obronną. Przeciwnicy poruszają się w określony sposób, sama wieża strzela automatycznie po określonym czasie. Najważniejszym aspektem tej gry jest sterowanie wieżą która może być poruszana na lewo i prawo.

\begin{lstlisting}[caption={Skrypt scr\_direction}]
if(point_direction(xstart,ystart,x,y) > 270 
   or point_direction(xstart,ystart,x,y) < 90){
    obj_tower.direction_s = 1;
}

if(point_direction(xstart,ystart,x,y) > 90 
   and point_direction(xstart,ystart,x,y) < 270){
    obj_tower.direction_s = 2;
}
\end{lstlisting}

Gdy użytkownik przesunie palcem po ekranie, gra zapisuje współrzędne początku oraz końca ruchu. Potem porównuje te dwa punkty, sprawdzając pod jakim kątem znajdują się one od siebie. Następnie zależnie od kąta przesunięcia palca, wieża może zmienić swój kierunek. Powyżej zawarty skrypt ukazuje sprawdzenie kąta przesunięcia oraz wyboru w którą stronę powinna przemieszczać się wieża.

\subsection{Rozbij jajko}
Rozbicie jajka polega na stuknięciu w nie odpowiednią ilość razy.  Zawsze gdy otrzyma odpowiednią ilość uderzeń coraz bardziej pęka, zmieniając się wizualnie.

\begin{lstlisting}[caption={Fragment  kodu obiektu obj\_egg}]
if (current_hit == 0){
    if(image_index &lt; 4){
        image_index += 1;
        current_hit = hit_count;
    } else{
        image_index += 1;
        depth = -150
        scr_end_win();
    }
}
\end{lstlisting}

Powyższy kod ukazuje sprawdzenie warunku zwycięstwa przy każdym kolejnym uderzeniu. Jeśli została spełniona ilość ciosów, należy dodać złamań. Jeśli użytkownik rozbije w końcu jajko, nastąpi ostatnia zmiana na obraz stłuczonego jajka oraz ekran końca gry.


\subsection{Powódź}
Ta gra składa się na używanie sześciu przycisków, które zmieniają kolor, dotychczas pochłoniętych kwadratów.  Po pokolorowaniu kwadratów które posiadają wartość "flood\_flooded\[i,n\] \= true", program sprawdza wszystkie przyległe pola(nie po skosie) w poszukiwaniu takiego samego koloru na jaki się pomalowało resztę. W przypadku kiedy znajdzie takie sam kolor, wchłania go zmieniając parametr flood\_flooded z false na true i rekurencyjnie sprawdza dla tego nowego kwadratu, czy on też nie znajduje się obok innych o tym samym kolorze, które nie zostały jeszcze pochłonięte.

\begin{lstlisting}[caption={Fragment  skryptu scr\_retrue\_colors}]
//checking up
if(i-1 >= 0 && obj_flood.flood_color[i-1,n] == arg_color){
    if(obj_flood.flood_flooded[i-1,n] != true){     
        obj_flood.flood_flooded[i-1,n] = true;
        scr_retrue_colors(i-1,n,arg_color);
    }   
}
\end{lstlisting}

Powyższy fragment ukazuje sprawdzenie czy nad aktualnie pochłoniętym kwadratem nie znajduje się taki o tym samym kolorze, analogicznie wyglądają sprawdzenia dla kwadratu pod spodem, jak i z prawej i lewej strony. Oczywiście nie sprawdzamy dalszą rekurencją tych kwadratów które zostały już pochłonięte.


\subsection{Uratuj księżniczkę}
W tej grze, gracz musi unikać spotkań z wrogiem aby jak najszybciej dostać się do księżniczki. Aby stworzyć iluzję nieskończonego poruszania się do przodu, to nie bohater się porusza lecz jego przeciwnicy. W przypadku znalezienia się na brzegu planszy, każdy przeciwnik zostaje zwrócony na drugi brzeg, wraz z losowym torem na którym się pojawia, tak jak demonstruje to poniższy kod.

\begin{lstlisting}[caption={Fragment kodu obiektu obj\_orc}]
if(x &lt; 120){
    y = choose(750, 1000, 1250);
    x = 960;
    obj_hero.win_count -= 1;
}
\end{lstlisting}

Jeśli bohater trafi na przeciwnika to przestaje na chwile biec aby go pokonać. Jeśli trafi na za dużą liczbę przeciwników, skończy mu się czas i nie zdąży dobiec do księżniczki.

\subsection{Zator}
Zator polega na odsunięciu z drogi wszystkich klocków które zasłaniają wyjście zielonemu blokowi. Problemem jest fakt że wszystkie klocki można przesuwać tylko wzdłuż ich dłuższej krawędzi. Zważając na założenia tej gry, najważniejszym jej aspektem jest sprawdzanie kolizji pomiędzy klockami i zapobieganie ewentualnemu nakładaniu się na siebie elementów. 

\begin{lstlisting}[caption={Fragment kodu obiektu obj\_green\_one}]
if(!(place_meeting(mouse_x + xx, y, obj_vertical_one)
      or place_meeting(mouse_x + xx, y, obj_horizontal_one)) 
      and (mouse_x+xx &lt;850 and mouse_x+xx &gt; 230)){
        if(grab == false){
            exit;
        } else{
            x = mouse_x + xx;
        }    
}
\end{lstlisting}

Jak widać na załączonym powyżej fragmencie obiektu zielonego bloku, w przypadku poruszenia jakimkolwiek blokiem, program wpierw sprawdza czy wykonany ruch nie spowoduje kolizji i dopiero po upewnieniu się że nic nie stoi na drodze, przesuwa blok tak jak było to zaplanowane.


\subsection{Lodowy Labirynt}
Założeniem Lodowego Labiryntu jest to że gdy zaczniemy się poruszać, nie możemy zmienić kierunku dopóki nie zatrzymamy sie na ścianie. Tym samym aplikacja została zaprogramowana tak aby sprawdzić w którą stronę przesunęliśmy palcem, co wyznacza kierunek ruchu. W momencie rozpoczęcia ruchu, blokujemy skrypt odpowiedzialny za wybieranie kierunku.

\begin{lstlisting}[caption={Fragment kodu obiektu obj\_player\_ice}]
    if(place_meeting(x-speed_ice, y, obj_lab_wall)){
        while(!place_meeting(x-1, y, obj_lab_wall)){
            x -= 1;
        }
\end{lstlisting}

W przytoczonym fragmencie widzimy jak co klatkę jest sprawdzane położenie obiektu i czy nie styka się już ze ścianą. Jest to fragment sprawdzający kolizję w przypadku ruchu w lewo. Dla każdego kierunku kod ten różni się ze względu na inne położenie na osiach.


\subsection{Dziurawy labirynt}
W przeciwieństwie do innych zadań, w tym nie sterujemy za pomocą ruchu palca lecz za pomocą przechylania urządzenia. Program na bieżąco bada czy i która oś jest przekręcona i dostosowuje do tego ruch kulą. W przypadku wpadnięcia w dziurę poziom sie resetuje. 

\begin{lstlisting}[caption={Fragment kodu obiektu obj\_player\_ball}]
if(place_meeting(x+xx, y, obj_lab_wall)){
    while(!place_meeting(x-device_get_tilt_x(), y, obj_lab_wall)){
        x += -device_get_tilt_x();
    }
}else{
    x += xx;
}
\end{lstlisting}

 Takie zachowanie wymagało innego algorytmu kolizji ponieważ ważnym było aby kula przesuwała się po krawędzi w przypadku przechylenia urządzania w jej stronę. Tak jak to pokazuje powyższy przykład dla osi X. 


\subsection{Super Pamięć}
Na koniec przewidziana jest gra w super pamięć. Wykorzystane zostały grafiki z poprzednich zadań. Gdy gracz odkrywa pierwszą kartę, zapamiętane zostaje jaki obrazek odkrył. Gdy odkrywa drugą kartę, program sprawdza zgodność przedstawionego obrazu z poprzednim wybranym. Po tym sprawdzeniu decyduje się na jedną z dwóch akcji.

\begin{lstlisting}[caption={Fragment kodu obiektu obj\_memory\_card}]
    if(obj_memory_shuffle.now_turned == 1){
        obj_memory_shuffle.now_shown = obj_memory_shuffle.card_deck[| id_fac];
        obj_memory_shuffle.now_shown_id = id;
    } else if(obj_memory_shuffle.now_turned == 2){
        if(obj_memory_shuffle.now_shown == obj_memory_shuffle.card_deck[| id_fac]){
            alarm[0] = obj_memory_shuffle.delay_time;
        }else{
            alarm[1] = obj_memory_shuffle.delay_time;
        }
    }
\end{lstlisting}

W przypadku kiedy rysunki się ze sobą zgadzają, wywoływany zostaje alarm który powoduje usunięcie obu instancji odkrytych kart. W drugim przypadku, gdy karty się ze sobą nie zgadzają, wywoływany jest alarm powodujący zakrycie kart z powrotem. Niezależnie od tego który alarm zostanie wywołany, informacje o ostatnio odkrytym zostają zresetowane.


\section{Ankieta}
Na potrzeby zakończenia badania, została zaimplementowana prosta ankieta w której użytkownik ma za zadanie wpisać tytuły oraz krótkie opisy zadań które wykonywał. Ankieta składa się na dwa pola tekstowe oraz możliwość wpisania kolejnego zadania lub zakończenia wpisywania zadań.  Tutaj nie ma żadnego limitu czasowego, użytkownik ma wpisać to co pamięta i jak to pamięta. Wszystkie te dane, jak i informację o tym które zadania zdołał ukończyć zapisują sie na serwerze, dla każdego użytkownika w osobnym folderze.






% zakończenie
\summary

PODSUMOWANIE TUTAJ

% załączniki (opcjonalnie):
%\appendix
%\chapter{Tytuł załącznika jeden}

%Treść załącznika jeden.

%\chapter{Tytuł załącznika dwa}

%Treść załącznika dwa.

% literatura (obowiązkowo):
\bibliographystyle{unsrt}
\bibliography{xml}
\begin{thebibliography}{9}

\bibitem{Zeigarnik}
Bluma Zeigarnik, „On Finished and Unfinished Tasks”\\
\textit{\url{http://codeblab.com/wp-content/uploads/2009/12/On-Finished-and-Unfinished-Tasks.pdf/}}

\bibitem{GMS}
GameMaker: Studio Dokumentacja”\\
\textit{\url{https://docs.yoyogames.com//}}(Dostęp: 25.05.2017)

\bibitem{GMServer}
GameMaker Server Dokumentacja\\
\textit{\url{http://gamemakerserver.com/en/docs//}}(Dostęp: 25.05.2017)

\end{thebibliography}

% spis rysunków (jeżeli jest potrzebny):
\listoffigures

\lstlistoflistings 

\oswiadczenie

\end{document}
